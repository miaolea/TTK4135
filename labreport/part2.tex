\newcommand{\texMacro}[2]{\texttt{\textbackslash{#1}\{#2\}}}
\section{General LaTeX tips}\label{sec:latex_tips}
Some tips were given in \Cref{sec:intro}, and this section will elaborate with some more concrete examples.


\subsection{Citations and Reference Management}
In academic writing, it is very important to cite your sources. In Latex this is done by defining an an entry in a \emph{BibTeX} bibliography file like this (from \texttt{bibliography.bib}):
\lstinputlisting[language=Tex, firstline=1, lastline=7]{bibliography.bib}
and then using the \texttt{\textbackslash{cite}} command in your Latex document. For instance \texttt{\textbackslash{cite}\{Chen2014\}} will produce~\cite{Chen2014}.

There are many different citation styles, and a lot of customization that is possible, so please check out e.g.~\cite{BiberBibtexEtc,WikibookLatex}\footnote{Keep citation of web pages to a minimum, and consider using \url{http://web.archive.org} if you are worried that the reference may change or be removed in the future.}.

There is also a lot of useful software to manage your references. Some popular examples include JabRef (\url{http://www.jabref.org/}), Mendeley (\url{https://www.mendeley.com/}) and EndNote. JabRef is perhaps the simplest of these three, and stores all information in a \texttt{.bib} file that you can directly use in your Latex document. Both Mendeley and EndNote can export references as BibTeX.
